\documentclass[muchmore,11pt]{article}%
\usepackage{amssymb}
\usepackage{amsfonts}
\usepackage{graphicx}
\usepackage{amsmath}
\usepackage{named}
\usepackage{flexcite}
\usepackage{zeros}
\usepackage{comment}
\usepackage{litprog}%
\setcounter{MaxMatrixCols}{30}
%TCIDATA{OutputFilter=latex2.dll}
%TCIDATA{Version=5.50.0.2960}
%TCIDATA{CSTFile=Literate Programming Article.cst}
%TCIDATA{Created=Friday, December 17, 2004 15:26:57}
%TCIDATA{LastRevised=Thursday, August 11, 2016 17:29:52}
%TCIDATA{<META NAME="GraphicsSave" CONTENT="32">}
%TCIDATA{<META NAME="SaveForMode" CONTENT="1">}
%TCIDATA{BibliographyScheme=BibTeX}
%TCIDATA{<META NAME="DocumentShell" CONTENT="Articles\Literate Programming Article">}
%TCIDATA{Language=American English}
%BeginMSIPreambleData
\providecommand{\U}[1]{\protect\rule{.1in}{.1in}}
%EndMSIPreambleData
\newtheorem{theorem}{Theorem}
\newtheorem{acknowledgement}[theorem]{Acknowledgement}
\newtheorem{algorithm}[theorem]{Algorithm}
\newtheorem{axiom}[theorem]{Axiom}
\newtheorem{case}[theorem]{Case}
\newtheorem{claim}[theorem]{Claim}
\newtheorem{conclusion}[theorem]{Conclusion}
\newtheorem{condition}[theorem]{Condition}
\newtheorem{conjecture}[theorem]{Conjecture}
\newtheorem{corollary}[theorem]{Corollary}
\newtheorem{criterion}[theorem]{Criterion}
\newtheorem{definition}[theorem]{Definition}
\newtheorem{example}[theorem]{Example}
\newtheorem{exercise}[theorem]{Exercise}
\newtheorem{lemma}[theorem]{Lemma}
\newtheorem{notation}[theorem]{Notation}
\newtheorem{problem}[theorem]{Problem}
\newtheorem{proposition}[theorem]{Proposition}
\newtheorem{remark}[theorem]{Remark}
\newtheorem{solution}[theorem]{Solution}
\newtheorem{summary}[theorem]{Summary}
\newenvironment{proof}[1][Proof]{\textbf{#1.} }{\ \rule{0.5em}{0.5em}}
\begin{document}

\title{Proof-Outline Logic.\\(Sequential Programming Edition)}
\author{Theodore S\ Norvell\\Electrical and Computer Engineering\\Memorial University}
\date{Draft typeset
%TCIMACRO{\TeXButton{Today}{\today}}%
%BeginExpansion
\today
%EndExpansion
}
\maketitle

\begin{abstract}
An introduction to proof outlines and Hoare logic.

\end{abstract}

\textbf{Note on editions:}\ This is the Sequential Programming edition,
designed to support MUN course Engi-6892 Algorithms:\ Correctness and
Complexity.\ There is also a Concurrent Programming edition, which contains
additional sections on concurrent programming. The Concurrent Programming
edition uses a somewhat different notation ---one that better matches Andrew's
text book. \cite{Andrews2000}.

\section{Preface}

This note provides background on assertions and the use of assertions in
designing correct programs.

The ideas presented are mainly due, for the sequential programming part, to
Floyd \cite{Floyd670} and Hoare \cite{Hoare690}. Blikle \cite{Blikle1979}
presented an early version of proof-outline logic.

Sections \ref{assertions} and \ref{seq} deal with sequential programming. As
such they are an elaboration of Hoare's excellent `Axiomatic basis' paper
\cite{Hoare690}. I\ suggest reading Hoare's paper first. For sequential
programs, proof-outline logic is just like Hoare logic except that commands
contain internal assertions.

\section{Assertions and logic\label{assertions}}

\subsection{Conditions and assertions}

A \textbf{condition} is a boolean expression with free variables chosen from
the state variables of a program. For example if we have variables

\begin{code}
int x ;

int y ;
\end{code}

Then the following are all examples of conditions%
\begin{align*}
x  &  <y\\
x  &  =y\\
x+y  &  =0\\
x  &  \geq0\\
x  &  \geq0\wedge x+y=0\qquad\text{.}%
\end{align*}


A condition that is expected to be true every time execution passes a
particular point in a program is called an \textbf{assertion}. In this course,
assertions are preceded by \#\# and are followed by an end-of-line like
this:\footnote{This fragment uses both the symbol $:=$ and $=$.This is one of
those \textquotedblleft notational improvements\textquotedblright%
\ I\ mentioned. I will use $:=$ for assignment and either $=$ or $==$ for
equality when writing pseudo-code. In C, C++, and Java, $=$ is used for
assignment and $==$ for equality. Andrews follows the C/C++/Java convention.
If you want to be on the safe side of any possible misunderstanding, you can
use $:=$ for assignment and $==$ for equality.}

\begin{code}
int x ;

int y ;

x := 5 ;

y := -5 ;

\#\#\ $x\geq0\wedge x+y=0$

z := x+z ;
\end{code}

Sometimes I'll write assertions inside curly brackets like this:

\begin{code}
int x ; int y ; x := 5 ; y := -5 ; $\left\{  x\geq0\wedge x+y=0\right\}  $ z
:= x+y ;
\end{code}

\subsection{Assertions in C, C++, and Java}

If one is programming in C or C++, then assertions may be written either as
comments or using the assert macro from the standard C\ library. E.g.

\begin{code}
\#include
%TCIMACRO{\TEXTsymbol{<}}%
%BeginExpansion
$<$%
%EndExpansion
assert.h%
%TCIMACRO{\TEXTsymbol{>}}%
%BeginExpansion
$>$%
%EndExpansion


...

\textbf{int} x ;

\textbf{int} y ;

x = 5 ;

y = -5 ;

assert(\ x
%TCIMACRO{\TEXTsymbol{>}}%
%BeginExpansion
$>$%
%EndExpansion
=0 \&\&\ x+y == 0 )\ ;
\end{code}

\noindent Assertions written using the \textsf{assert} macro will be evaluated
at run time and the program will come to a grinding halt, should the assertion
ever evaluate to false.\footnote{By using a different include file, one can,
of course, make the action followed on a false assertion be whatever you like.
For example, in a desktop application, one might cause all files to be saved
and an error report to be assembled and e-mailed back to the developers; in an
embedded system, one might cause the system to go into safe mode. If you
program in C or C++, I\ strongly suggest redefining the assert macro in a way
that suits your application. And use it!}

In Java one easily create one's own \textsf{Assert} class with a
\textsf{check} method in it.\footnote{As of Java 1.4, there is actually an
\textsf{assert} keyword in Java. However I don't recommend its use. Assertion
checking is turned off by default in most (if not all) JVMs. This can be
compared to removing the seat belts from a car's design once it goes into
production. In my own work I use my own assertion checking class. I\ recommend
you do the same.}

\begin{code}
\textbf{public} \textbf{class} Assert \{

\begin{indent}
\item \textbf{public static void} check( \textbf{boolean} b ) \{

\begin{indent}
\item if( !b ) \{ \textbf{throw} \textbf{new} java.lang.AssertionError() ; \}
\end{indent}

\item \}
\end{indent}

\}
\end{code}

\noindent This can be used in your code as follows:

\begin{code}
\textbf{int} x ;

\textbf{int} y ;

x = 5 ;

y = -5 ;

Assert.check(\ x
%TCIMACRO{\TEXTsymbol{>}}%
%BeginExpansion
$>$%
%EndExpansion
=0 \&\&\ x+y == 0 )\ ;
\end{code}

\subsubsection{Be an assertive programmer}

Using assertions has several benefits.

\begin{itemize}
\item In the design process, they help you articulate what conditions you
expect to be true at various points in program execution.

\item In testing, executable assertions can help you identify errors in your
code or in your design.

\item In execution, executable assertions ---combined with a recovery
mechanism--- can help make your program more fault tolerant.

\item Assertions provide valuable documentation. Executable assertions are
more valuable than comments, as they are more likely to be accurate.
\end{itemize}

Whether to code assertions as comments or as executable checks is a question
that depends on the local conditions of the project you are working on.
Sometimes it has to be answered on a case-by-case basis. In this course we
will concentrate on the use of assertions in the design process, rather than
on their (nevertheless important)\ uses in testing, documentation, and in
making systems fault-tolerant. My general advice is to make assertions
executable as much as is practical.\footnote{In concurrent programming there
is an additional complication in making assertions executable, namely that
they should be evaluated atomically. Consider the assertion%
\[
x=0\vee y=0
\]
if we evaluate this in parallel with the following sequence of assignments%
\[
x:=0;y:=1;
\]
it is possible that the assertion will evaluate to false even though there is
no time at which it is in fact false.
\par
This problem can be solved by evaluating assertions only when the thread has
exclusive access to the data they refer to.}

\subsection{Substitutions}

Sometimes it is useful to create a new condition by replacing all free
occurrences of a variable $x$ in a condition $P$ by an expression $(E)$. We
write $P[x:E]$ for the new condition.\footnote{A variety of notations are used
by authors for substitutions. I\ like this one because it doesn't use
subscripts or superscripts.} For example%
\[%
\begin{tabular}
[c]{lll}%
$\left(  x\geq0\wedge x+y=0\right)  [x:z]$ & is & $\left(  z\right)
\geq0\wedge\left(  z\right)  +y=0$\\
$\left(  x\geq0\wedge x+y=0\right)  [x:x+y]$ & is & $\left(  x+y\right)
\geq0\wedge\left(  x+y\right)  +y=0$\\
$\left(  2y=5\right)  [y:y+z]$ & is & $2(y+z)=5\qquad\text{.}$%
\end{tabular}
\ \ \ \ \ \ \ \ \
\]
It is useful to extend this notation to allow simultaneous substitution for
more than one variable. For example%
\[
\left(  x\geq0\wedge x+y=0\right)  [x,y:z,x]\text{ is }\left(  z\right)
\geq0\wedge\left(  z\right)  +\left(  x\right)  =0
\]
Usually we omit the parentheses in contexts where they are not required.

\subsection{Propositional and predicate logic}

In this section, I will review a little bit of propositional and predicate
logic. We use notations $\lnot$ (not), $=$ (equality), $\wedge$ (and), $\vee$,
(or), and $\Rightarrow$ (implication). Precedence between the operators is in
the same order. Some of the laws of propositional logic that will be useful
are%
\[%
\begin{tabular}
[c]{ll}%
$\left(  P\Rightarrow Q\right)  =\left(  \lnot P\vee Q\right)  $ & Material
implication\\
$\left(  \mathit{true}\Rightarrow P\right)  =P$ & Identity\\
$\mathit{false}\Rightarrow P$ & Antidomination\\
$P\Rightarrow P$ & Reflexivity\\
$P\Rightarrow\mathit{true}$ & Domination\\
$\left(  P\Rightarrow(Q\Rightarrow R)\right)  =\left(  P\wedge Q\Rightarrow
R\right)  $ & Shunting\\
$\left(  P0\Rightarrow Q\right)  \Rightarrow\left(  P0\wedge P1\Rightarrow
Q\right)  $ & Subsetting the antecedent\\
$\left(  P\Rightarrow Q\wedge R\right)  =\left(  P\Rightarrow Q\right)
\wedge\left(  P\Rightarrow R\right)  $ & Distributivity
\end{tabular}
\ \ \ \
\]


Also frequently useful are the \textbf{one-point laws}, which let you make use
of information from equalities. $E$ and $F$ range over expressions of any
type, $v$ is a variable of that same type.
\begin{align*}
\left(  E=F\Rightarrow P[v:E]\right)   &  =\left(  E=F\Rightarrow
P[v:F]\right) \\
\left(  E=F\wedge P[v:E]\right)   &  =\left(  E=F\wedge P[v:F]\right)
\end{align*}
To see that these are true, consider the case where $E=F$ and then the case
where $E\neq F$.

For example%
\[
x=X\wedge y=Y\Rightarrow x^{y}=X^{Y}%
\]
simplifies, using one-point (and shunting), to%
\[
x=X\wedge y=Y\Rightarrow X^{Y}=X^{Y}%
\]
which then simplifies to%
\[
x=X\wedge y=Y\Rightarrow\mathit{true}%
\]
which is then true.

Any condition that is true for all assignments of values to its free
variables, is called a \textbf{universally true} formula. For example,
(assuming $x$ and $y$ and $z$ are integer variables) the following are all
universally true%
\begin{align*}
&  \left.  2+2=4\right. \\
&  \left.  x<y\wedge y<z\Rightarrow x<z\right. \\
&  \left.  x+1>x\right.
\end{align*}
However, $x^{2}+y^{2}=z^{2}$ is not universally true, because there is an
assignment for which it is not true; for example $3^{2}+4^{2}=6^{2}$ is not true.

Whether a condition is universally true may depend on the types ascribed to
its variables. For example, if we are using Java and $x$ has type
\textbf{int}, then, by the rules of the Java language, the value of $x+1,$
when $x$ is $2^{31}-1,$ is $-2^{31}$; so $x+1>x$ is not universally true when
$x$ represents a Java \textbf{int} and $+$ is interpreted as Java \textbf{int} addition.

\begin{comment}
    Note that Dafny uses short circuiting three valued logic, so it might be a good idea to change POL to
    do the same.
\end{comment}

Sometimes expressions are undefined, for example, supposing $x$ and $y$ are
rational variables, $x/y$ is undefined when $y$ is $0$. This raises the
question of whether $1=x/x$ is universally true. We'll say that such an
expression is not universally true. However, the expression $x\neq
0\Rightarrow1=x/x$ is universally true; if we consider the case of $x=0$ , we
have $\mathrm{false}\Rightarrow?$, and applying the principle of
antidomination, this is $\mathrm{true}$. Using $?$ to represent an unknown or
undefined truth value, we can fill in truth tables for the propositional logic
as follows%
\begin{align*}
&
\begin{tabular}
[c]{|l||l|}\hline
$\lnot$ & \\\hline\hline
false & true\\\hline
? & ?\\\hline
true & false\\\hline
\end{tabular}
\ \qquad%
\begin{tabular}
[c]{|l||l|l|l|}\hline
$\wedge$ & false & ? & true\\\hline\hline
false & false & false & false\\\hline
? & false & ? & ?\\\hline
true & false & ? & true\\\hline
\end{tabular}
\\
&
\begin{tabular}
[c]{|l||l|l|l|}\hline
$\vee$ & false & ? & true\\\hline\hline
false & false & ? & true\\\hline
? & ? & ? & true\\\hline
true & true & true & true\\\hline
\end{tabular}
\ \qquad%
\begin{tabular}
[c]{|l||l|l|l|}\hline
$\Rightarrow$ & false & ? & true\\\hline\hline
false & true & true & true\\\hline
? & ? & ? & true\\\hline
true & false & ? & true\\\hline
\end{tabular}
\end{align*}


\section{Sequential programming\label{seq}}

\subsection{Contracts}

A specification for a component indicates the operating conditions (i.e. the
conditions under which the component is expected to operate) and the function
of the component (i.e. the relationship between the component's inputs and
outputs). For example we might specify a resistor by saying that the
relationship between the voltage and current across the resistor is given by%
\[
953I\leq V\leq1050I\qquad\text{,}%
\]
provided%
\[
0\leq V\leq+10\qquad\text{.}%
\]
The latter formula gives the operating conditions, the former the relation
between inputs and outputs.

In programming, we can use a pair of assertions as a specification or
\textquotedblleft contract\textquotedblright\ for a command or subroutine. The
first assertion is the so-called precondition, it specifies the operating
conditions, that is, the state of the program when the command begins
operation. The second assertion specifies the state of the program when (and
if) the command ends operation. For example, the following pair of conditions
specifies a command that results in $x$ being assigned the value 5, provided
that $y$ is initially $4$%
\[
\lbrack y=4,x=5]\qquad\text{,}%
\]
where $x$ and $y$ are understood to be state variables of type \textsf{int}.
One solution to this particular contract is

\begin{code}
$\left\{  y=4\right\}  $

$x:=y+1$

$\left\{  x=5\right\}  \qquad$.
\end{code}

\noindent Such a triple, consisting of a precondition, a command, and a
postcondition, is called a \textbf{Hoare triple} after C.A.R. Hoare, who
introduced the idea to programming. Here is another solution:

\begin{code}
$\left\{  y=4\right\}  $

$x:=5$ ;

$\left\{  x=5\right\}  \qquad$.
\end{code}

\noindent Here is one more:

\begin{code}
$\left\{  y=4\right\}  $

$y:=y+1$ ;

$x:=y$ ;

$\left\{  x=5\right\}  \qquad$.
\end{code}

\noindent Nothing in the contract says that $y$ must not change!

If you do want to specify that a variable does not change, then `constants'
can be used. Constants are conventionally written with capital letters. The
following contract specifies that, provided $y$ is initially less than 100,
$y$ must not change and the final value of $x$ must be larger than that of
$y$:%
\[
\lbrack y<\mathrm{100}\wedge y=Y,y=Y\wedge x>y]
\]
where it is understood that $x$ and $y$ are state variables of type
\textsf{int},\textsf{ }while $Y$ is a constant\footnote{The word `constant' is
the traditional term to use. In the mathematical sense, $Y$ is a variable. We
use the term `constant' to distinguish such mathematical variables from
\textquotedblleft program variables\textquotedblright\ which refer to
components of the program state. The point is that $Y\ $can't be changed by
the execution of the program so $Y$ represents the same value in both the
precondition and in the postcondition. Since the precondition implies $y=Y$
and the postcondition implies $y=Y$, it is clear that $y$ has the same value
in the final state as it has in the initial state.
\par
It is so common to use constants equated to the initial values of variables
that the following convention has evolved. The notation $e_{o}$ is used to
refer to the value of expression $e$ in the original state. Thus this contract
could be written as%
\[
\left[  y<100,y=y_{o}\wedge x>y\right]
\]
} of type \textsf{int}.

\subsection{Partial correctness}

Consider a Hoare triple $\{P\}\;S\;\{Q\}$. We define that the triple is
\textbf{partially correct} if and only if, for all possible values of all
constants, whenever the execution of $S$ is started in a state satisfying $P$,
the execution of $S$ does not crash and can only end in a state satisfying $Q$.

Note that the definition of partial correctness does not require that $S$
should terminate. Thus the following triple is partially correct even if we
interpret $x$ to have the type $\mathbb{Z}$, that is, to range over all
mathematical integers%
\[
\{\mathrm{true}\}\quad\mathbf{while}\;x\neq0\;\mathbf{do}%
\ x:=x-1\;\mathbf{end\;while}\quad\{x=0\}\qquad\text{.}%
\]
If we consider initial states where $x$ is positive or zero, then eventually
the while-loop will terminate and the program will halt in a state where
$x=0$, satisfying the postcondition. If we start the while-loop in an initial
state where $x$ is negative, then the while-loop will never terminate and so
execution \textquotedblleft can only end in a state
satisfying\textquotedblright\ the postcondition by virtue of the fact it never
ends at all!

We write $\vdash\{P\}\;S\;\{Q\}$ to mean \textquotedblleft$\{P\}\;S\;\{Q\}$ is
partially correct\textquotedblright

In concurrent programming, we are often interested in processes that do not
terminate (e.g., in embedded systems) so dealing with partial correctness is
an appropriate and desirable thing to do. If termination is important, we can
deal with it as a separate concern. From here on, we won't be worried about
any other kind of correctness, so we will just say \textquotedblleft
correct\textquotedblright.

\subsection{Some examples of assignments and a rule}

Here are some small examples of Hoare triples. In each case the variables
should be understood to be integers%

\[
\{x+1=y\}\;x:=x+1\;\{x=y\}
\]
Is this triple correct?\ (Answer for yourself before reading on...) If
initially $y$ is $x+1$ and we change $x$ to $x+1$, then finally both $x$ and
$y$ will equal the original value of $x+1$, and so they will equal each other.
Yes, it is correct.

How about%
\[
\{2x=3y\}\;x:=2x\;\{x=3y\}
\]
Is this correct? Well, if initially $2x$ is $3y$, then, after changing $x$ to
$2x$, finally $x$ will be $3y$.

These two examples suggest a general rule, which is%
\[
\vdash\left\{  Q[x:E]\right\}  \;x:=E\;\left\{  Q\right\}
\]
When $Q$ is the postcondition of an assignment $x:=E$, we call $Q[x:E]$ the
\textbf{substituted postcondition}. More generally, it is sufficient\ for the
precondition to imply $Q[x:E]$, so a better rule is%
\[
\vdash\left\{  P\right\}  \;v:=E\;\left\{  Q\right\}  \text{ exactly if
}P\Rightarrow Q[v:E]\text{ is universally true,}%
\]
where $v$ is any variable, $E$ is any expression, and $P$ and $Q$ are any
conditions. Here is an example where the precondition is stronger than it
needs to be:%
\[
\left\{  2x<3y\right\}  \;x:=2x\;\left\{  x\leq3y\right\}
\]
The substituted postcondition is $(x\leq3y)[x:2x]$, which is $2x\leq3y$. By
the assignment rule, this triple is correct exactly if $2x<3y\Rightarrow
2x\leq3x$ is universally true.

There is one more aspect to assignment that should be mentioned. This is that
the expression might not always be well defined. For example, if we divide by
$0$, this is an error and we should consider that if this happens the program
has crashed. Since a command that crashes is not partially correct we should
really ensure that our rule for assignments includes checking that the
expression is well defined. Let's suppose that for each expression $E$ there
is a condition $\mathrm{df}[E]$ that says that $E$ is well-defined, i.e. does
not crash when evaluated. For example $\mathrm{df}[x/y]$ might be $y\neq0$.
Now the improved assignment rule is
\begin{equation}%
\begin{tabular}
[c]{ll}%
$\vdash\left\{  P\right\}  \;v:=E\;\left\{  Q\right\}  $ & $\text{exactly if
}P\Rightarrow Q[v:E]\text{ is universally true}$\\
& $\text{and }P\Rightarrow\mathrm{df}[E]\text{ is universally true}$%
\end{tabular}
\ \text{ }\tag{assignment rule}%
\end{equation}
In many cases $\mathrm{df}[E]$ is simply $\mathrm{true}$, and so it is trivial
that $P\Rightarrow\mathrm{df}[E]$ is universally true.

This rule generalizes to simultaneous assignments to multiple variables. For
example%
\[
\left\{  x<y\right\}  \;x,y:=y,x\;\left\{  y\leq x\right\}
\]
The substituted postcondition is $\left(  y\leq x\right)  [x,y:y,x]$, which is
$x\leq y$; this is implied (for all values of $x$ and $y$) by $x<y$.

Here is one last example of an assignment; it will be of use later.%
\[
\left\{  y\geq0\wedge x=X\wedge y=Y\right\}  \;z:=1\;\left\{  y\geq0\wedge
X^{Y}=z\times x^{y}\right\}
\]
First we find the substituted postcondition%
\[
y\geq0\wedge X^{Y}=1\times x^{y}%
\]
which simplifies to
\[
y\geq0\wedge X^{Y}=x^{y}%
\]
This (using one-point laws) is implied by the precondition $y\geq0\wedge
x=X\wedge y=Y$.

\subsection{A bigger example}

Here is another example. I claim that
\begin{equation}
\left\{  y\geq0\wedge x=X\wedge y=Y\right\}  \;S\;\left\{  z=X^{Y}\right\}
\qquad\text{,} \label{a}%
\end{equation}
is correct, where
\begin{align*}
S  &  \triangleq\left(  z:=1;\mathbf{while(}\;y>0\;)\;T\right) \\
T  &  \triangleq\mathbf{if}\;\mathit{\mathrm{odd}}(y)\;\mathbf{then}%
\;U\;\mathbf{else}\;V\;\mathbf{end\;if}\\
U  &  \triangleq(z:=z\times x;y:=y-1;)\\
V  &  \triangleq(x:=x\times x;y:=y/2;)\qquad\text{.}%
\end{align*}


To show that this triple is correct, we'll need to deal with constructs other
than assignments. For that we introduce a new idea:\ proof outlines.

\subsection{Proof outlines}%

%TCIMACRO{\TeXButton{B Fig}{\begin{figure}[tb]}}%
%BeginExpansion
\begin{figure}[tb]%
%EndExpansion


\begin{code}
$\left\{  y\geq0\wedge x=X\wedge y=Y\right\}  $

$z:=1$ ;

$\left\{  y\geq0\wedge X^{Y}=z\times x^{y}\right\}  $

\textbf{while}\ $y>0$ \textbf{do}

\begin{indent}
\item $\left\{  X^{Y}=z\times x^{y}\wedge y>0\right\}  $

\item \textbf{if}\ \textrm{$odd$}$(y)$ \textbf{then}

\begin{indent}
\item $\left\{  X^{Y}=z\times x^{y}\wedge y>0\wedge\mathrm{odd}(y)\right\}  $

\item $z:=z\times y$

\item $\left\{  y-1\geq0\wedge X^{Y}=z\times x^{y-1}\right\}  $

\item $y:=y-1$
\end{indent}

\item \textbf{else}

\begin{indent}
\item $\left\{  X^{Y}=z\times x^{y}\wedge y>0\wedge\mathrm{even}(y)\right\}  $

\item $x:=x\times x$

\item $\left\{  \mathrm{even}(y)\wedge y/2\geq0\wedge X^{Y}=z\times
x^{y/2}\right\}  $

\item $y:=y/2$
\end{indent}

\item \textbf{end if}
\end{indent}

\thinspace\textbf{end\ while}

$\left\{  z=X^{Y}\right\}  $
\end{code}

%

%TCIMACRO{\TeXButton{E Fig}{\caption{An example proof outline.}\label
%{po}\end{figure}}}%
%BeginExpansion
\caption{An example proof outline.}\label{po}\end{figure}%
%EndExpansion


A \textbf{proof outline} is a command that is annotated with assertions. It
represents the outline of a proof of the program. Figure \ref{po} is a proof
outline for the example of the last section.

A proof outline is not a proof: it is (if correct)\ a summary of a proof. This
is why it is called a `proof outline'.

\subsection{Correctness of proof outlines}

We can formally define \textbf{partially correct} proof outlines for
sequential programs as follows:

\textbf{Assignment Rule:} $\left\{  P\right\}  \;v:=E\;\left\{  Q\right\}  $
is a partially correct proof outline if
\[
P\Rightarrow Q[v:E]\text{ is universally true and }P\Rightarrow\mathrm{df}%
[E]\text{ is universally true\qquad.}%
\]


\textbf{Skip Rule:} $\left\{  P\right\}  \;\mathbf{skip}\;\left\{  Q\right\}
$ is a partially correct proof outline if%
\[
P\Rightarrow Q\text{ is universally true\qquad.}%
\]


\textbf{(Sequential) Composition Rule:} $\left\{  P\right\}  \;S\;\left\{
Q\right\}  \;T\;\left\{  R\right\}  $ is a partially correct proof outline, provided

\begin{itemize}
\item that $\left\{  P\right\}  \;S\;\left\{  Q\right\}  $ is a partially
correct proof outline, and

\item that $\left\{  Q\right\}  \;T\;\left\{  R\right\}  $ is a partially
correct proof outline.
\end{itemize}

\textbf{2-Tailed If Rule}: $\left\{  P\right\}  \;\mathbf{if(}\;E\;)\;\{Q_{0}%
\}\ S\;\mathbf{else}\;\{Q_{1}\}\;T\;\left\{  R\right\}  $ is a partially
correct proof outline, provided

\begin{itemize}
\item that $\{Q_{0}\}\;S\;\{R\}$ is a partially correct proof outline,

\item that $\{Q_{1}\}\;T\;\{R\}$ is a partially correct proof outline,

\item that $P\Rightarrow\mathrm{df}[E]$ is universally true,

\item that $P\wedge E\Rightarrow Q_{0}$ is universally true, and

\item that $P\wedge\lnot E\Rightarrow Q_{1}$ is universally true.
\end{itemize}

\textbf{1-Tailed If Rule:} $\left\{  P\right\}  \;\mathbf{if(}%
\;E\;)\;\{Q\}\;S\;\left\{  R\right\}  $ is a partially correct proof outline, provided

\begin{itemize}
\item that $\{Q\}\;S\;\{R\}$ is a partially correct proof outline,

\item that $P\Rightarrow\mathrm{df}[E]$ is universally true,

\item that $P\wedge E\Rightarrow Q$ is universally true, and

\item that $P\wedge\lnot E\Rightarrow R$ is universally true.
\end{itemize}

\textbf{Iteration Rule:} $\left\{  P\right\}  \;\mathbf{while(}%
\;E\;)\;\{Q\}\;S\;\left\{  R\right\}  $ is a partially correct proof outline, provided

\begin{itemize}
\item that $P\Rightarrow\mathrm{df}[E]$ is universally true,

\item that $P\wedge E\Rightarrow Q$ is universally true,

\item that $P\wedge\lnot E\Rightarrow R$ is universally true, and

\item that $\{Q\}\;S\;\{P\}$ is a partially correct proof outline.
\end{itemize}

By the way, the loop's precondition, $P$, is called an \textbf{invariant} of
the loop. Loop invariants are crucial in designing and documenting loops. Note
that, provided it is true when the while command starts, the invariant will be
true at the start of each iteration and when the loop terminates. Loop
invariants allow us to analyze the effect of a loop by considering only the
effect of a single iteration.

From here on we will write \textquotedblleft\textbf{correct}\textquotedblright%
\ in place of \textquotedblleft partially correct\textquotedblright, as we
won't be concerned with any other sort of correctness.

In a proof outline, all commands will be preceded by an assertion. This is its
\textbf{precondition}. In the example, the precondition of $y:=y/2;$ is
\[
\mathrm{even}(y)\wedge y/2\geq0\wedge X^{Y}=z\times x^{y/2}%
\]
and the precondition of $x:=x\times x;$ is%
\[
X^{Y}=z\times x^{y}\wedge y>0\wedge\mathrm{even}(y)\text{\qquad.}%
\]


Suppose $\{P\}\;S$\ $\{Q\}$ is a correct proof outline. Let $\widehat{S}$ be
formed by deleting all assertions from $S$ or by treating them as comments.
Now $\{P\}\;\widehat{S}\;\{Q\}$ is a correct Hoare triple.

\subsection{Correctness of the example}

There is a little, but not much, work left to show that the example proof
outline in Figure \ref{po} is correct. First a recall that\ a boolean formula
is said to be universally true if it evaluates to true regardless of the
values chosen for its free variables (including our so-called constants).

Let's call the loop invariant $I$.%
\[
I\triangleq y\geq0\wedge X^{Y}=z\times x^{y}%
\]


\begin{itemize}
\item Because of the first assignment, we must show%
\[
y\geq0\wedge x=X\wedge y=Y\Rightarrow I[z:1]
\]
is universally true. After substitution we have%
\[
y\geq0\wedge x=X\wedge y=Y\Rightarrow y\geq0\wedge X^{Y}=1\times x^{y}%
\]
which (using a one-point law) we can easily see is universally true.

\item From the rule for while-loops, we must show%
\begin{align*}
&  I\wedge y>0\Rightarrow X^{Y}=z\times x^{y}\wedge y>0\\
&  I\wedge\lnot\left(  y>0\right)  \Rightarrow z=X^{Y}%
\end{align*}
are each universally true. Both are fairly straight-forward

\item From the rule for 2-tailed if commands, we must show%
\begin{align*}
X^{Y}  &  =z\times x^{y}\wedge y>0\wedge\mathrm{odd}(y)\Rightarrow
X^{Y}=z\times x^{y}\wedge y>0\wedge\mathrm{odd}(y)\text{ and}\\
X^{Y}  &  =z\times x^{y}\wedge y>0\wedge\lnot\mathrm{odd}(y)\Rightarrow
X^{Y}=z\times x^{y}\wedge y>0\wedge\mathrm{even}(y)
\end{align*}
are each universally true. Both are trivial.

\item The four assignments in the loop body give rise to four expressions that
should be shown to be universally true:%
\begin{align*}
&  \left.  X^{Y}=z\times x^{y}\wedge y>0\wedge\mathrm{odd}(y)\right.
\Rightarrow\left(  y-1\geq0\wedge X^{Y}=z\times x^{y-1}\right)  [z:z\times
y]\\
&  \left.  y-1\geq0\wedge X^{Y}=z\times x^{y-1}\right.  \Rightarrow I[y:y-1]\\
&  \left.  X^{Y}=z\times x^{y}\wedge y>0\wedge\mathrm{even}(y)\right.
\Rightarrow\left(  \mathrm{even}(y)\wedge y/2\geq0\wedge X^{Y}=z\times
x^{y/2}\right)  [x:x\times x]\\
&  \left.  \mathrm{even}(y)\wedge y/2\geq0\wedge X^{Y}=z\times x^{y/2}\right.
\Rightarrow I[y:y/2]
\end{align*}

\end{itemize}

\bibliographystyle{named}
\bibliography{big}



\end{document}